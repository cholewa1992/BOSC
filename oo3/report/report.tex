\documentclass[danish]{report}

\usepackage[utf8]{inputenc}
\usepackage[danish]{babel}
\usepackage{listings}
\usepackage{color}
\usepackage{courier}
\usepackage{parskip}
\usepackage{graphicx}
\usepackage{amsfonts}

\definecolor{dkgreen}{rgb}{0,0.6,0}
\definecolor{gray}{rgb}{0.5,0.5,0.5}
\definecolor{mauve}{rgb}{0.58,0,0.82}

\lstset{
  frame=,
  language=C,
  aboveskip=3mm,
  belowskip=3mm,
  showstringspaces=false,
  columns=flexible,
  basicstyle={\small\ttfamily},
  numbers=none,
  numberstyle=\tiny\color{gray},
  keywordstyle=\color{blue},
  commentstyle=\color{dkgreen},
  stringstyle=\color{mauve},
  breaklines=true,
  breakatwhitespace=true
  tabsize=4
}

% Title Page
\title{Obligatorisk opgave 3}
\author{Jacob B. Cholewa \& Mathias Pedersen }

\begin{document}
\maketitle

\chapter{Introduktion}
I denne rapport arbejdes med tredje obligatoriske opgave i kurset Operativsystemer og C. I denne opgave arbejder vi med koncepter som Macroer, linkedlists og moduler i linuxkernen.

\chapter{Metode (Mathias)}
I opgaven er stillet en række problemstillinger som skal løses på bedste vis. Alle opgaverne har fokus på C kode, som vi vil afvikle på en Ubuntu Linux virtuel boks. De fleste opgaver vil kræve implementering, og disse implementeringer vil blive forklared i længden af denne rapport. Vores løsninger testes under implementeringen på logisk basis og derefter ved observering af det afviklede programs opførsel.

\chapter{Teori}
\section{Linux kernen og moduler (Mathias)}

Linux kerne moduler er kodestykker som dynamisk kan loades og unloades en kørende linux kerne. De skal have en \textit{module\_init} og \textit{module\_exit} metode, som hhv. afvikles når modulet loades og unloades. Da moduler er del af kernen kan de bruge system kald direkte.

\section{Macro metoder i C}

I C har man mulighed for at definere macro metoder. En macro er et stykke kode fragment som man knytter til et macro navn. Når navnet bruges i koden vil preprocessoren indsætte kode fragmentet. Nedenfor ses macroen \texttt{min} og hvordan preprocessoren indsætter kodefragmentet ved brug.
\begin{lstlisting}
#define min(X, Y)  ((X) < (Y) ? (X) : (Y))
x = min(a, b);          // x = ((a) < (b) ? (a) : (b));
y = min(1, 2);          //y = ((1) < (2) ? (1) : (2));
z = min(a + 28, *p);    //z = ((a + 28) < (*p) ? (a + 28) : (*p));
\end{lstlisting}

\section{Taskstruct og Linkedlist i Linuxkernen}

Biblioteket \texttt{<linux/list.h>} er en linkedlist implimentation i Linux. Implementationen bygger på structen \texttt{list\_head}. Structen indeholder en pointer til det næste og forrige element og på den måde kan man lave en liste af elementer.

\begin{lstlisting}
struct list_head {
    struct list_head *next, *prev;
};
\end{lstlisting}

Til at traversere listen kan man bruge list\_for\_each\_entry macroen

\begin{lstlisting}
/**
 * list_for_each_entry  -       iterate over list of given type
 * @pos:        the type * to use as a loop cursor.
 * @head:       the head for your list.
 * @member:     the name of the list_struct within the struct.
 */
#define list_for_each_entry(pos, head, member) \
    for (pos = list_entry((head)->next, typeof(*pos), member); \
        prefetch(pos->member.next), &pos->member != (head); \
            pos = list_entry(pos->member.next, typeof(*pos), member))
\end{lstlisting}

Macroen tager, givet det første element, og løber listen i gennem ved hele tiden af tage elementets next, altså det næste element, indtil enden af listen af nået.

En af de andre biblioteker der benytter linkedlist implementationen er \texttt{<linux/sched.h>}. En del af \texttt{sched} er \texttt{task\_struct}. Denne struct bruges til at holde information om processer der kører på systemmet. Structen indeholder information om processens børneprocess. Derfor kan vi traversere gennem process træet og finde alle processer der køre på systemmet.

\begin{lstlisting}
struct task_struct {
    ... fields ...
    struct list_head children;      /* list of my children */
    struct list_head sibling;       /* linkage in my parent's children list */
}
\end{lstlisting}

For at kunne traversere processtræet har vi brug for den process som er forældre til alle andre processer i systemet. Denne process, init processen, er gemt i \texttt{init\_task} feltet i \texttt{<linux/sched.h>}

\section{Dybde først søgning}
DFS, eller Dybde først søgning, er en algoritme til at løbe en graf igennem. Algoritmen består af en stak. Hver gang man tager et element af stakken ligger man alle de ikke allerede besøgt elementer man kan nå fra elementet på stakken. Derefter forsættes processen indtil der ikke er flere elementer på stakken. På det tidspunkt vil man have besøgt alle elementer.

Da C er et stak sprog behøver man ikke et stak da program flowet naturligt vil følge et staks opførsel.  


\chapter{Implementation}
\section{Simple mod (Jacob)}

Dette eksempel er et linux modul som indeholder personers fødselsdag og printer fødselsdagene til \textit{kernel output}. Vi har altså derfor brug for en struct til at holde information om personer fødselsdag.

\begin{lstlisting}
typedef struct birthday {
        int day;
        int month;
        int year;
        struct list_head list;
} Birthday;
\end{lstlisting}

I vores module init metode opretter vi derfor en liste personer. Vi bruger kmalloc til at allokere hukommelse til strucen og fylder den derefter med personens data. Derefter itererer vi ved hjælp af \texttt{list\_for\_each\_entry} igennem den liste vi lige har oprettet og printer alle personernes fødselsdag til \textit{kernel output}.

Når modulet bliver fjernet igen kaldes modulets exit metode hvor vi igen itererer over listen for at fjerne personerne og deallokerer hukommelsen igen. 


\section{Tasklist mod (Mathias)}
I denne opgave skal vi meget som den forrige iterere over en linked list og udskrive information. I denne opgave drejer det sig om kørende processor på systemet. Opgaven lyder på at løbe gennem processorne med en dybde først søgning og udskrive visse informationer om dem. Al udprintning sker til \textit{kernel output} med \textit{printk()} funktionen.

For at kunne lave en dybde først søgning har vi oprettet en metode som vil bruges rekursivt. Metoden hedder \textit{dfs} og koden ses herunder.

\begin{lstlisting}
void dfs(struct task_struct *tsk, struct list_head *ptr){
	//Struct used for holding current task
	struct task_struct *task;
	
	//For each children of the tsk task
	list_for_each(ptr, &tsk->children){
		
		//Takes the task at the current position in the list
		task = list_entry(ptr, struct task_struct, sibling);
		
		//Prints information about the task
		printk(KERN_INFO "%d %s\n", task->pid, task->comm);
		
		//Call resurcivly on the current task
		dfs(task,list);
	}
}
\end{lstlisting}

Metoden tager to parametre.

\begin{itemize}
	\item[*tsk] er en pointer til et task element
	\item[*ptr] er en pointer til elementets faders 
\end{itemize}

\chapter{Verificering (Jacob)}
\chapter{Konklusion (Mathias)}
\chapter{Referencer}

\end{document}          

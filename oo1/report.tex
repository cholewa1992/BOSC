\documentclass[danish]{report}

\usepackage[utf8]{inputenc}
\usepackage[danish]{babel}
\usepackage{listings}
\usepackage{color}
\usepackage{courier}
\usepackage{parskip}

\definecolor{dkgreen}{rgb}{0,0.6,0}
\definecolor{gray}{rgb}{0.5,0.5,0.5}
\definecolor{mauve}{rgb}{0.58,0,0.82}

\lstset{
  frame=,
  language=C,
  aboveskip=3mm,
  belowskip=3mm,
  showstringspaces=false,
  columns=flexible,
  basicstyle={\small\ttfamily},
  numbers=none,
  numberstyle=\tiny\color{gray},
  keywordstyle=\color{blue},
  commentstyle=\color{dkgreen},
  stringstyle=\color{mauve},
  breaklines=true,
  breakatwhitespace=true
  tabsize=4
}

% Title Page
\title{Obligatorisk opgave 1}
\author{Jacob B. Cholewa, Rasmus L. Wismann \& Nikolai Storr }


\begin{document}
\maketitle
\newpage
\tableofcontents
\newpage

\chapter{Introduktion}
Denne rapport omhandler implementationen af vores shell BOSC. Specifikationerne er som følger.


\section{Features}
\begin{enumerate}
\item bosh skal kunne virke uafhængigt. Du må ikke bruge andre eksisterende shells, f.eks. er det ikke tilladt at anvende et systemkald {\tt system()} til at starte {\tt bash}.
\item En bruger skal kunne indtaste almindelige enkeltstående kommandoer, så som {\tt ls}, {\tt cat} og {\tt wc}. Hvis kommandoen ikke findes i operativ systemet skal der udskrives en ”Command not found“ meddelelse.
\item Kommandoer skal kunne eksekvere som baggrundsprocesser (ved brug af \&) såadan at mange programmer kan køres på samme tid.
\item Der skal være indbygget funktionalitet som gør de muligt at lave redirection af stdin og stdout til filer. F.eks skal kommandoen {\tt wc -l < /etc/passwd > antalkontoer} lave en fil ”antalkontoer“, der indeholder antallet af brugerkontoer.
\item Det skal være muligt at anvende pipes. F.eks. skal {\tt ls | wc -w} udskrive antallet af filer.
\item Funktionen {\tt exit} skal være indbygget til at afslutte shell’en.
\item Tryk på Ctrl-C skal afslutte det program, der kører i {\tt bosh} shell’en, men ikke shell’en selv.
\end{enumerate}

Følgende vil handle om vores implementation af disse features 


\chapter{Implementation og teori}
\section{Feature 1}

Vi håndterer brugerens input uden brug af andre shells. F.e.ks bruger vi ikke {\tt system()} til at starte {\tt bash} eller til andre system kald. 


\section{Feature 2}

For at en bruger kan bruge kommandoerne {\tt ls}, {\tt cat} og {\tt wc} skal vi kigge i de forskelle {\tt bin} arkiver på vores Ubuntu installation. Vi leder i mapperne {\tt ./}, {\tt /bin/} og {\tt /usr/bin/}. Dette gør {\tt execvp} dog selv, men vi checker stadig til at starte med at filerne findes i disse mapper. Hvis filen er blevet fundet returneres {\tt 1} ({\tt true}), hvis ikke returneres {\tt 0} ({\tt false}). Hvis kommandoen {\tt exit} blev fundet returneres {\tt -1}. Kode stykket ses her under. Status koderne bliver tolket til eventuelle fejlbeskeder i vores executeshellcmd metode.

\begin{lstlisting}
int isValidCmd(char **cmd){
    if(strncmp(*cmd,"exit",4) == 0) return -1;

    char str1[100];
    char str2[100];
    char str3[100];

    char *path1 = "/bin/";
    char *path2 = "/usr/bin/";
    char *path3 = "./";

    strcpy(str1,path1);
    strcat(str1, *cmd);
    strcpy(str2,path2);
    strcat(str2, *cmd);
    strcpy(str3,path3);
    strcat(str3, *cmd);

    return 
        access ( str1, F_OK ) != -1 || 
        access ( str2, F_OK ) != -1 || 
        access ( str3, F_OK ) != -1;
}
\end{lstlisting}


\section{Feature 3}

Hvis en kommando køres med symbolet {\tt \&} skal processen startes som en baggrundsprocess. I shellcmd structen vil feltet {\tt background} være sat til {\tt 1} ({\tt true}) hvis processen skal køres som en baggrundsproces. Som det kan ses i kode stykket nedenfor venter vi derfor kun på barne processen hvis {\tt background} er {\tt 0} ({\tt false}).
\begin{lstlisting}
int executeshellcmd(Shellcmd *shellcmd){
    

    ... code to check shellcmd

    pid_t pid = fork();
    if(pid == 0){
           ... Code executing shellcmd
    }else{
        if(shellcmd -> background == 0){
            int wstatus = 0;    
            waitpid(pid,&wstatus,0);
        }
    }
    return 0;
}
\end{lstlisting}

\section{Feature 4}

Hvis symbolet {\tt >} optræder i kommandoen skal vi omdirigere output til filnavnet på venstre side. eg. {\tt cmd > file}. På samme måde skal vi hvis symbolet {\tt <} optræder i kommandoen omdirigere input til at være filen fra venstre side. eg {\tt cmd < file}. Dette har vi løst med følgende logik.

\begin{lstlisting}
if(in == NULL && out == NULL){
                status = executecmd(cmdlist);
            }else{
                if(in != NULL && out != NULL) status = redirInOut(in, out, cmdlist);
                if(in != NULL) status = redirIn(in, cmdlist);
                if(out != NULL) status = redirOut(out, cmdlist);
            }
\end{lstlisting}

som bruger følgende hjælpe metoder.

\begin{lstlisting}
// redirect in and out
int redirInOut(char *inFile, char *outFile, Cmd *cmdlist){
    
    int fidIn  = open(inFile, O_RDONLY);
    int fidOut = open(outFile, O_WRONLY | O_CREAT | O_APPEND);              
    close(0); close(1);
    dup(fidIn); dup(fidOut);
    int status = executecmd(cmdlist);
    close(fidIn); close(fidOut);
    if(status != 0) 
        printf("execvp returned: %i, errno returned: %i 'no such file or directory' \n", status, errno);

}

// redirect in
int redirIn(char *inFile, Cmd *cmdlist){
    int fid = open(inFile, O_RDONLY);  
    close(0); // close standard input
    dup(fid);   // 'duplicate fileid', opens another input (file)    
    int status = executecmd(cmdlist);
    close(fid);             
    if(status != 0) 
        printf("execvp returned: %i, errno returned: %i 'no such file or directory' \n", status, errno);


}

// redirect out
int redirOut(char *outFile, Cmd *cmdlist){
    int fid = open(outFile, O_WRONLY | O_CREAT | O_APPEND);         
    close(1); // close standard output
    dup(fid); // duplicate file-descriptor

    int status = executecmd(cmdlist);
    close(fid);             
    if(status != 0) 
        printf("execvp returned: %i, errno returned: %i 'no such file or directory' \n", status, errno);

}
\end{lstlisting}





\section{Feature 5}

En pipe (skrives med symbol {\tt |}) tager outputtet fra kommandoen på højre side og bruger det som input til kommandoen på venstre side. eg {\tt c1 | c2 | c3 } tager output fra c1 og giver til c2 som input osv. Vi har i vores shell implementeret dette med følgende logik. Denne metode kaldes af de metoder som var vist i feature 4.

\begin{lstlisting}
int executecmd(Cmd *cmdlist){
    int status;
    char **cmd = cmdlist -> cmd;
    Cmd *next = cmdlist -> next;
    if(next != NULL){

        int fd[2];
        pipe(fd);

        pid_t pid = fork();
        if(pid == 0){
            close(fd[0]);
            close(1);
            dup(fd[1]);
            close(fd[1]);
            status = executecmd(next);
        }else{
            close(fd[1]);
            close(0);
            dup(fd[0]);
            close(fd[0]);
            status = execvp(*cmd,cmd);
            
            int wstatus;
            waitpid(pid,&wstatus,0);
        }
    }else{
        status = execvp(*cmd,cmd);
    }
    return status;
}

\end{lstlisting}

\section{Feature 6}

%Denne funktion er allerede forklaret som en del af feature 6. 

I 'executeshellcmd' samler vi alle commands og sender dem til funktionen 'isValidCmd' for validering (unødvendigt, da execute funktionerne allerede returnerer -1 hvis en command ikke findes). Som sagt under Feature 2 returnerer denne funktion -1 Hvis 'exit' er en command. 'executeShellCmd' returnerer så umiddelbart 1 til 'main' metoden. Dermed terminerer while loopet i 'main' og vores shell slutter. 

I executeShellCmd() samler vi alle commands og sender dem til validering. Hvis status er -1 returnerer vi fra funktionen.

\begin{lstlisting}
while( cmdlist != NULL){
	char **cmd = cmdlist -> cmd;
	cmdlist = cmdlist -> next;
	status = isValidCmd(cmd);

	if(status == 0){ 
		printf("%s: command not found\n", *cmd );
		return 0;
	}
	if(status == -1) return 1;
}
\end{lstlisting}

I isValidCmd() returnerer vi -1 hvis 'exit' er en command.
\begin{lstlisting}
int isValidCmd(char **cmd){
	
	if(strncmp(*cmd,"exit",4) == 0) return -1;
\end{lstlisting}

I main(). Terminering af while loopet afslutter vores shell. 
\begin{lstlisting}
/* parse commands until 'exit' command, then exit */
while (!terminate) {
	terminate = executeshellcmd(&shellcmd);
\end{lstlisting}

\section{Feature 7}

Denne feature nåede ikke at blive implementeret i vores {\tt BOSH} shell.

\chapter{Konklusion}
\end{document}          
